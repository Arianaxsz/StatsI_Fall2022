\documentclass[12pt,letterpaper]{article}
\usepackage{graphicx,textcomp}
\usepackage{natbib}
\usepackage{setspace}
\usepackage{fullpage}
\usepackage{color}
\usepackage[reqno]{amsmath}
\usepackage{amsthm}
\usepackage{fancyvrb}
\usepackage{amssymb,enumerate}
\usepackage[all]{xy}
\usepackage{endnotes}
\usepackage{lscape}
\newtheorem{com}{Comment}
\usepackage{float}
\usepackage{hyperref}
\newtheorem{lem} {Lemma}
\newtheorem{prop}{Proposition}
\newtheorem{thm}{Theorem}
\newtheorem{defn}{Definition}
\newtheorem{cor}{Corollary}
\newtheorem{obs}{Observation}
\usepackage[compact]{titlesec}
\usepackage{dcolumn}
\usepackage{tikz}
\usetikzlibrary{arrows}
\usepackage{multirow}
\usepackage{subcaption}
\usepackage{xcolor}
\newcolumntype{.}{D{.}{.}{-1}}
\newcolumntype{d}[1]{D{.}{.}{#1}}
\definecolor{light-gray}{gray}{0.65}
\usepackage{url}
\usepackage{listings}
\usepackage{color}

\definecolor{codegreen}{rgb}{0,0.6,0}
\definecolor{codegray}{rgb}{0.5,0.5,0.5}
\definecolor{codepurple}{rgb}{0.58,0,0.82}
\definecolor{backcolour}{rgb}{0.95,0.95,0.92}

\lstdefinestyle{mystyle}{
	backgroundcolor=\color{backcolour},   
	commentstyle=\color{codegreen},
	keywordstyle=\color{magenta},
	numberstyle=\tiny\color{codegray},
	stringstyle=\color{codepurple},
	basicstyle=\footnotesize,
	breakatwhitespace=false,         
	breaklines=true,                 
	captionpos=b,                    
	keepspaces=true,                 
	numbers=left,                    
	numbersep=5pt,                  
	showspaces=false,                
	showstringspaces=false,
	showtabs=false,                  
	tabsize=2
}
\lstset{style=mystyle}
\newcommand{\Sref}[1]{Section~\ref{#1}}

\title{ Problem Set 1 Response}
\date{October 1st, 2022}
\author{Ariana Antunes}

\begin{document}
	\maketitle
	
\section*{Question 1 (50 points): Education}

\subsection*{Find 90\% confidence interval for the average student IQ in the school.}

\lstinputlisting[language=R, firstline=56, lastline=68]{AA_PS01.R}  

\begin{verbatim}
	    Lower    Upper 
	    94.13283 102.7472
	    
	    mean(y)[1] 98.44

\end{verbatim}

\subsection*{The hypothesis test with $\alpha=0.05$.} 

\lstinputlisting[language=R, firstline=80, lastline=85]{AA_PS01.R}  

\begin{verbatim}
	One Sample t-test 
	
	data: y
	t = 37.593, df = 24, p-value < 2.2e-16
	alternative hypothesis: true mean is not equal to 0
	
	5 percent confidence interval: 98.27407 98.60593
	sample estimates:mean of x     
	98.44 
	
\end{verbatim}

\newpage


\section{Question 2 (50 points): Political Economy}

\subsection*{The correlation plot between Y, X1, X2 and X3} 

\noindent You can also save figures in R, and place them in your answers that you're writing in your .tex file. First, you need to make sure your path/file name is correct, then you'll save your work when your in R (see code below).

\lstinputlisting[language=R, firstline=118, lastline=119]{AA_PS01.R}  
\vspace{.25cm}
\noindent With our figure saved, we just need to render it in our .tex file, which we can do using the \texttt{figure} environment:


\begin{figure}[h!]\centering
	\caption{\footnotesize Correlation between Y, X1, X2 and X3.}
	\label{fig:plot_1}
	\includegraphics[width=.75\textwidth]{Correlation_plot_expenditure.png}
\end{figure}

It seems the correlation appears to be much similar when compering the different variables. 

\subsection*{The correlation plot between Y and Region}


\begin{figure}[h!]\centering
	\caption{\footnotesize correlation plot  between Y and Region.}
	\label{fig:plot_1}
	\includegraphics[width=.75\textwidth]{plot_Y+Region.png}
\end{figure}

On average west region have the highest per capita expenditure on housing assistance. 


\subsection*{The correlation plot between Y, X1 and Region}


\lstinputlisting[language=R, firstline=137, lastline=139]{AA_PS01.R}  


\begin{figure}[h!]\centering
	\caption{\footnotesize correlation plot  between Y and Region.}
	\label{fig:plot_1}
	\includegraphics[width=.75\textwidth]{Correlation Y X1 & Region.png}
\end{figure}



\end{document}
